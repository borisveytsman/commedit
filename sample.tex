\documentclass{book}
\usepackage[papersize={204mm,275mm}, top=17mm,
bottom=17mm, outer=22mm, inner=20mm, includeheadfoot=false]{geometry}
\usepackage{kantlipsum}
\usepackage{commedit}
\clearpage
\begin{commeditPreamble}{commented.tex}
  \documentclass{book}
  \usepackage[papersize={230mm,288mm}, top=15mm,
  bottom=17mm, outer=11mm, inner=4.5mm, includeheadfoot=false]{geometry}
  \usepackage{kantlipsum}
  \usepackage{ragged2e}
  \usepackage{commedit}
  \basepageargs{width=140mm}
  \commentsHook{\RaggedRight\parskip=0pt\parindent=1em\clubpenalty=0
  \widowpenalty=0\relax}
\end{commeditPreamble}

\begin{commeditText}
  \frontmatter
  \title{Commented edition}
  \author{Boris Veytsman}
  \date{November 2018}
  \maketitle
  
\chapter{Introduction}
\label{sec:intro}

This is the introuduction for the commentaries
\kant[6-8]

\mainmatter
\end{commeditText}

\begin{document}
\frontmatter
\title{Base edition}
\author{Boris Veytsman}
\date{November 2018}
\maketitle

\chapter{Foreword}
\label{chap:foreword}


\kant[1-5]

\begin{commeditText}
  \kant[11-17]

  Some equation in the comments
  \begin{equation}
    \label{eq:pi}
    e^{i\pi}=-1
  \end{equation}
  and another one
  \begin{equation}
    \label{eq:sin}
    sin^2\phi+\cos^2\phi=1
  \end{equation}

\end{commeditText}

\mainmatter

\chapter{Some thoughts}
\label{chap:thoughts}



\begin{commeditComments}
  \kant[6]
\end{commeditComments}


Einstein equation
\begin{equation}
  \label{eq:Einstein}
  e=mc^2
\end{equation}

\kant[6]


\begin{commeditComments}
  \kant[2]

  We can reference base equation~(\ref{eq:Einstein}) on base
  page~(\pageref{eq:Einstein}) and commented edition
  equations~(\ref{eq:pi}) and (\ref{eq:sin}) on the commented edition
  page~(\pageref{eq:pi}).

  \kant[7]
\end{commeditComments}

\kant[5-8]

\begin{commeditComments}
  \kant[4]
\end{commeditComments}
\begin{commeditComments}
  \kant[8-20]
\end{commeditComments}

\kant[9-20]
\end{document}
